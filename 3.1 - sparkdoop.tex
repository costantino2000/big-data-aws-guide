\subsection{Configurazione di Hadoop e Spark}



\subsubsection{Configurazione di tutti i nodi}

\textcolor{red}{Per ogni istanza} si devono modificare quattro file di configurazione. Il primo, \textbf{hadoop-env.sh}, contiene la configurazione delle variabili di ambiente di Hadoop. Apriamolo:

\begin{verbatim}
    $ nano $HADOOP_CONF_DIF/hadoop-env.sh
\end{verbatim}

e sostituiamo la riga

\begin{verbatim}
    export JAVA_HOME=$JAVA_HOME
\end{verbatim}

con

\begin{verbatim}
    export JAVA_HOME=/usr/lib/jvm/java-8-openjdk-amd64
\end{verbatim}

Passiamo al secondo file, \textbf{core-site.xml}. Apriamolo:

\begin{verbatim}
    $ sudo nano $HADOOP_CONF_DIF/core-site.xml
\end{verbatim}

e inseriamo la seguente proprietà all'interno dell'elemento \textbf{\textlangle configuration\textrangle}, con al posto di HOSTNAME l'indirizzo Ipv4 \textbf{privato} del master (reperibile nella stessa schermata da cui si è preso quello pubblico):

\begin{verbatim}
    <property>
    <name>fs.defaultFS</name>
    <value>hdfs://HOSTNAME:9000</value>
    </property>
\end{verbatim}

Il terzo file, \textbf{yarn-site.xml}, serve a impostare su quale macchina girerà il source manager. Apriamolo:

\begin{verbatim}
    $ sudo nano $HADOOP_CONF_DIF/yarn-site.xml
\end{verbatim}

e, come fatto prima, inseriamo la seguente proprietà:

\begin{verbatim}
    <property>
    <name>yarn.nodemanager.aux-services</name>
    <value>mapreduce_shuffle</value>
    </property>
    <property>
    <name>yarn.resourcemanager.hostname</name>
    <value>namenode</value>
    </property>
\end{verbatim}

Infine, creiamo una copia del file \textbf{mapred-site.xml.template} e apriamolo:

\begin{verbatim}
    $ sudo cp $HADOOP_CONF_DIF/mapred-site.xml.template
        $HADOOP_CONF_DIF/mapred-site.xml
    $ sudo nano $HADOOP_CONF_DIF/mapred-site.xml
\end{verbatim}

per inserire la seguente proprietà:

\begin{verbatim}
    <property>
    <name>mapreduce.jobtracker.address</name>
    <value>namenode:54311</value>
    </property>
    <property>
    <name>mapreduce.framework.name</name>
    <value>yarn</value>
    </property>
\end{verbatim}



\subsubsection{Configurazione dell'istanza master}

Andiamo \textcolor{red}{nel terminale del master} e apriamo il file \textbf{hdfs-site.xml}:

\begin{verbatim}
    $ sudo nano $HADOOP_CONF_DIF/hdfs-site.xml
\end{verbatim}

per inserire la seguente proprietà (dentro il tag \textit{configuration} come fatto finora):

\begin{verbatim}
    <property>
    <name>dfs.replication</name>
    <value>2</value>
    </property>
    <property>
    <name>dfs.namenode.name.dir</name>
    <value>file:///home/ubuntu/hadoop/data/hdfs/namenode</value>
    </property>
    <property>
    <name>dfs.datanode.data.dir</name>
    <value>file:///home/ubuntu/hadoop/data/hdfs/datanode</value>
    </property>
\end{verbatim}

Creiamo ora le cartelle in cui andranno contenuti i dati del master e del nodo attivo sull'istanza master (datanode1):

\begin{verbatim}
    $ sudo mkdir -p $HADOOP_HOME/data/hdfs/namenode
    $ sudo mkdir -p $HADOOP_HOME/data/hdfs/datanode
    $ sudo nano $HADOOP_CONF_DIF/masters
\end{verbatim}

e scriviamoci dentro il nome del master:

\begin{verbatim}
    namenode
\end{verbatim}

Apriamo poi il file degli slave:

\begin{verbatim}
    $ sudo nano $HADOOP_CONF_DIF/slaves
\end{verbatim}

e scriviamoci i nomi degli slave (se contiene \textit{localhost} lo si può rimuovere senza problemi):

\begin{verbatim}
    datanode1
    datanode2
\end{verbatim}

Infine diamo all'utente \textit{ubuntu} la proprietà della cartella in cui si trova la home di Hadoop:

\begin{verbatim}
    $ sudo chown -R ubuntu $HADOOP_HOME
\end{verbatim}



\subsubsection{Configurazione dell'istanza datanode2}

Andiamo nel terminale \textcolor{red}{della seconda istanza} e apriamo il file \textbf{hdfs-site.xml}:

\begin{verbatim}
    $ sudo nano $HADOOP_CONF_DIF/hdfs-site.xml
\end{verbatim}

per inserire la seguente proprietà nella configurazione:

\begin{verbatim}
    <property>
    <name>dfs.replication</name>
    <value>2</value>
    </property>
    <property>
    <name>dfs.namenode.name.dir</name>
    <value>file:///home/ubuntu/hadoop/data/hdfs/namenode</value>
    </property>
    <property>
    <name>dfs.datanode.data.dir</name>
    <value>file:///home/ubuntu/hadoop/data/hdfs/datanode</value>
    </property>
\end{verbatim}

Creiamo poi nel nodo la cartella in cui andranno contenuti i dati dello slave:

\begin{verbatim}
    $ sudo mkdir -p $HADOOP_HOME/data/hdfs/datanode
\end{verbatim}

e diamo all'utente \textit{ubuntu} la proprietà della cartella in cui si trova la home di Hadoop:

\begin{verbatim}
    $ sudo chown -R ubuntu $HADOOP_HOME
\end{verbatim}




\subsubsection{Installazione e configurazione di Spark}

L'installazione di Spark va fatta \textcolor{red}{su ogni nodo del cluster}. Scarichiamo ed estraiamo
Spark spostandolo nella cartella \textbf{/home/ubuntu/spark}, e alla fine rimuoviamo il
file .tgz scaricato:

\begin{verbatim}
    $ wget https://archive.apache.org/dist/spark
        /spark-3.2.4/spark-3.2.4-bin-hadoop2.7.tgz
    $ tar xvzf spark-3.2.4-bin-hadoop2.7.tgz
    $ sudo mv ./spark-3.2.4-bin-hadoop2.7 /home/ubuntu/spark
    $ rm spark-3.2.4-bin-hadoop2.7.tgz
\end{verbatim}

Per impostare le variabili d'ambiente di Spark (che includono la configurazione di Hadoop) cloniamo e apriamo il file \textbf{spark-env.sh}:

\begin{verbatim}
    $ sudo cp spark/conf/spark-env.sh.template spark/conf/spark-env.sh
    $ sudo nano spark/conf/spark-env.sh
\end{verbatim}

e inseriamo le seguenti righe, sostituendo NAMENODEADDRESS con l'indirizzo privato del master:

\begin{verbatim}
    export SPARK_MASTER_HOST="NAMENODEADDRESS"
    export HADOOP_CONF_DIR="/home/ubuntu/hadoop/conf"
\end{verbatim}
